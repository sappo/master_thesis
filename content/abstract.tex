% !TEX root = ../thesis-example.tex
%
\pdfbookmark[0]{Zusammenfassung}{Zusammenfassung}
\chapter*{Zusammenfassung}
\label{sec:abstract}
\vspace*{-10mm}

Entity Resolution ist der Prozess, in einer oder mehreren Datenquellen
Gruppen von Datensätzen zu identifizieren, die derselben realen Entität
entsprechen. Dabei gibt es kein einzigartiges Attribut, welches zur
Zuordnung genutzt werden kann. Die Unterschiede in den Datensätzen
entstehen, beispielsweise durch Rechtschreibfehler oder fehlende und
vertauschte Attribute, welche eine Vieldeutigkeit erzeugen, die bei
manueller Betrachtung durch einen Menschen meist nur mit viel Zeitaufwand
aufzulösen sind. Damit ein Entity Resolution Workflow diese
Vieldeutigkeiten auslösen kann, muss dieser abhängig von der Domäne der
Daten konfiguriert werden. Diese Konfiguration besteht aus einer Vielzahl
von Parametern, die auch von einem Domänexperten nur aufwändig zu
bestimmen sind. Erster Beitrag dieser Arbeits ist deshalb die Analyse und
Entwicklung von Verfahren, die eine Selbstkonfiguration der Parameter in
Abhängigkeit der Datenquelle ermöglichen. Dabei liegt der Fokus dieser
Arbeit auf Entity Resolution Verfahren für Event Stream Processing
Systeme. Hierbei ist neben der Qualität der Ergebnisse auch die
Antwortzeit von Bedeutung, welche oft im Subsekundenbereich liegen muss.
Die Suche nach Duplikaten ist jedoch mit enormen Kosten verbunden, die
beim vollständigen Durchsuchen aller Bestandsdatensätze in einer
quadratischen Komplexität resultiert. Der Zweite Beitrag dieser Arbeit
ist daher ein sog. Blocking-Verfahren zur Reduzierung der Komplexität,
welches für Event Stream Processing tauglich ist. Für die
Selbstkonfiguration bedeutet dies, dass neben der Qualität auch die
Effizienz berücksichtigen muss. Die analysierten und entwickelten
Verfahren wurden in einem prototypischen System implementiert, dass sich
unüberwacht (ohne Eingreifen des Benutzers) vor der Laufzeit selbst
konfiguriert und anschließend Anfragen aus einem Ereignisstrom
beantwortet. Die Auswertung dieses Systems zeigt, dass die
Selbstkonfiguration auf einem Datensatz mit 4 Mio Einträgen ein F-Measure
von bis zu 70 \% erreicht und bei 1.3 Mio Anfragen im Durchschnitt über
500 pro Sekunde beantwortet.
