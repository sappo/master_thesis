% !TEX root = ../thesis-example.tex
%
\pdfbookmark[0]{Zusammenfassung}{Zusammenfassung}
\chapter*{Zusammenfassung}
\label{sec:abstract}
\vspace*{-10mm}

Entity Resolution ist der Prozess, in einer oder mehreren Datenquellen,
Gruppen von Datensätzen zu identifizieren, die derselben realen Entität
entsprechen. Dabei gibt es kein einzigartiges Attribut, welche zur
Zuordnung genutzt werden kann. Die Unterschiede in den Datensätzen sind,
beispielsweise Rechtschreibfehler oder fehlende und vertauschte Attribute,
welche eine Vieldeutigkeit erzeugen, die auch bei manueller Betrachtung
durch einen Menschen, meist nur mit viel Zeitaufwand und eventuellen
Zusatzinformationen, aufzulösen sind. Dabei liegt der Fokus dieser Arbeit
auf Entity Resolution Verfahren für Event Stream Processing Systeme.
Hierbei ist neben der Qualität der Ergebnisse auch die Antwortzeit von
Bedeutung, welche oft im Subsekundenbereich liegen muss. Diese Anforderung
ist herausfordernd, da die Suche nach Duplikaten mit enormen Kosten
verbunden sind, die bei einer vollständigen Suche in einer quadratischen
Komplexität resultiert. Ein Beitrag dieser Arbeit daher ein Verfahren zur
Reduzierung der Komplexität, welches für Event Stream Processing tauglich
ist. Damit dieses Verfahren die Anforderungen an Event Stream Processing
erfüllt, muss eine entsprechende Konfiguration bestimmt werden. Diese
Konfiguration ist allerdings von der Domäne der Daten abhängig und besteht
aus einer Vielzahl von Parametern, die auch von Domänexperten nur
aufwändig zu bestimmen ist. Deshalb wurden in dieser Arbeit Verfahren und
Algorithmen analysiert und entwickelt, die eine Selbstkonfiguration der
Parameter ermöglichen, die den Kompromiss zwischen Qualtiät und Effizienz
für ein Event Stream Processing System eingehen. Die analysierten und
entwickelten Verfahren wurden in einem prototypischen System
implementiert, dass sich vor der Laufzeit selbst konfiguriert und
anschließend Ereignisse aus einem Ereignisstrom beantwortet. Die
Auswertung dieses Systems zeigt, dass die Selbstkonfiguration qualitativ
gute Ergebnisse liefert und das System Anfragen effizient im
Subsekundenbereich beantwortet.
